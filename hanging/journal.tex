
%% bare_jrnl.tex
%% V1.3
%% 2007/01/11
%% by Michael Shell
%% see http://www.michaelshell.org/
%% for current contact information.
%%
%% This is a skeleton file demonstrating the use of IEEEtran.cls
%% (requires IEEEtran.cls version 1.7 or later) with an IEEE journal paper.
%%
%% Support sites:
%% http://www.michaelshell.org/tex/ieeetran/
%% http://www.ctan.org/tex-archive/macros/latex/contrib/IEEEtran/
%% and
%% http://www.ieee.org/



% *** Authors should verify (and, if needed, correct) their LaTeX system  ***
% *** with the testflow diagnostic prior to trusting their LaTeX platform ***
% *** with production work. IEEE's font choices can trigger bugs that do  ***
% *** not appear when using other class files.                            ***
% The testflow support page is at:
% http://www.michaelshell.org/tex/testflow/


%%*************************************************************************
%% Legal Notice:
%% This code is offered as-is without any warranty either expressed or
%% implied; without even the implied warranty of MERCHANTABILITY or
%% FITNESS FOR A PARTICULAR PURPOSE!
%% User assumes all risk.
%% In no event shall IEEE or any contributor to this code be liable for
%% any damages or losses, including, but not limited to, incidental,
%% consequential, or any other damages, resulting from the use or misuse
%% of any information contained here.
%%
%% All comments are the opinions of their respective authors and are not
%% necessarily endorsed by the IEEE.
%%
%% This work is distributed under the LaTeX Project Public License (LPPL)
%% ( http://www.latex-project.org/ ) version 1.3, and may be freely used,
%% distributed and modified. A copy of the LPPL, version 1.3, is included
%% in the base LaTeX documentation of all distributions of LaTeX released
%% 2003/12/01 or later.
%% Retain all contribution notices and credits.
%% ** Modified files should be clearly indicated as such, including  **
%% ** renaming them and changing author support contact information. **
%%
%% File list of work: IEEEtran.cls, IEEEtran_HOWTO.pdf, bare_adv.tex,
%%                    bare_conf.tex, bare_jrnl.tex, bare_jrnl_compsoc.tex
%%*************************************************************************

% Note that the a4paper option is mainly intended so that authors in
% countries using A4 can easily print to A4 and see how their papers will
% look in print - the typesetting of the document will not typically be
% affected with changes in paper size (but the bottom and side margins will).
% Use the testflow package mentioned above to verify correct handling of
% both paper sizes by the user's LaTeX system.
%
% Also note that the "draftcls" or "draftclsnofoot", not "draft", option
% should be used if it is desired that the figures are to be displayed in
% draft mode.
%
\documentclass[journal]{journal}
%
% If IEEEtran.cls has not been installed into the LaTeX system files,
% manually specify the path to it like:
% \documentclass[journal]{../sty/IEEEtran}

\usepackage{changepage}

\newenvironment{myindent}{\begin{adjustwidth}{1cm}{}}{\end{adjustwidth}}

\usepackage{listings}
\usepackage{color}

\definecolor{dkgreen}{rgb}{0,0.6,0}
\definecolor{gray}{rgb}{0.5,0.5,0.5}
\definecolor{mauve}{rgb}{0.58,0,0.82}

\lstset{frame=tb,
  language=Java,
  aboveskip=3mm,
  belowskip=3mm,
  showstringspaces=false,
  columns=flexible,
  basicstyle={\small\ttfamily},
  numbers=none,
  numberstyle=\tiny\color{gray},
  keywordstyle=\color{blue},
  commentstyle=\color{dkgreen},
  stringstyle=\color{mauve},
  breaklines=true,
  breakatwhitespace=true,
  tabsize=3
}


% Some very useful LaTeX packages include:
% (uncomment the ones you want to load)


% *** MISC UTILITY PACKAGES ***
%
%\usepackage{ifpdf}
% Heiko Oberdiek's ifpdf.sty is very useful if you need conditional
% compilation based on whether the output is pdf or dvi.
% usage:
% \ifpdf
%   % pdf code
% \else
%   % dvi code
% \fi
% The latest version of ifpdf.sty can be obtained from:
% http://www.ctan.org/tex-archive/macros/latex/contrib/oberdiek/
% Also, note that IEEEtran.cls V1.7 and later provides a builtin
% \ifCLASSINFOpdf conditional that works the same way.
% When switching from latex to pdflatex and vice-versa, the compiler may
% have to be run twice to clear warning/error messages.






% *** CITATION PACKAGES ***
%
%\usepackage{cite}
% cite.sty was written by Donald Arseneau
% V1.6 and later of IEEEtran pre-defines the format of the cite.sty package
% \cite{} output to follow that of IEEE. Loading the cite package will
% result in citation numbers being automatically sorted and properly
% "compressed/ranged". e.g., [1], [9], [2], [7], [5], [6] without using
% cite.sty will become [1], [2], [5]--[7], [9] using cite.sty. cite.sty's
% \cite will automatically add leading space, if needed. Use cite.sty's
% noadjust option (cite.sty V3.8 and later) if you want to turn this off.
% cite.sty is already installed on most LaTeX systems. Be sure and use
% version 4.0 (2003-05-27) and later if using hyperref.sty. cite.sty does
% not currently provide for hyperlinked citations.
% The latest version can be obtained at:
% http://www.ctan.org/tex-archive/macros/latex/contrib/cite/
% The documentation is contained in the cite.sty file itself.






% *** GRAPHICS RELATED PACKAGES ***
%
\ifCLASSINFOpdf
  % \usepackage[pdftex]{graphicx}
  % declare the path(s) where your graphic files are
  % \graphicspath{{../pdf/}{../jpeg/}}
  % and their extensions so you won't have to specify these with
  % every instance of \includegraphics
  % \DeclareGraphicsExtensions{.pdf,.jpeg,.png}
\else
  % or other class option (dvipsone, dvipdf, if not using dvips). graphicx
  % will default to the driver specified in the system graphics.cfg if no
  % driver is specified.
  % \usepackage[dvips]{graphicx}
  % declare the path(s) where your graphic files are
  % \graphicspath{{../eps/}}
  % and their extensions so you won't have to specify these with
  % every instance of \includegraphics
  % \DeclareGraphicsExtensions{.eps}
\fi
% graphicx was written by David Carlisle and Sebastian Rahtz. It is
% required if you want graphics, photos, etc. graphicx.sty is already
% installed on most LaTeX systems. The latest version and documentation can
% be obtained at:
% http://www.ctan.org/tex-archive/macros/latex/required/graphics/
% Another good source of documentation is "Using Imported Graphics in
% LaTeX2e" by Keith Reckdahl which can be found as epslatex.ps or
% epslatex.pdf at: http://www.ctan.org/tex-archive/info/
%
% latex, and pdflatex in dvi mode, support graphics in encapsulated
% postscript (.eps) format. pdflatex in pdf mode supports graphics
% in .pdf, .jpeg, .png and .mps (metapost) formats. Users should ensure
% that all non-photo figures use a vector format (.eps, .pdf, .mps) and
% not a bitmapped formats (.jpeg, .png). IEEE frowns on bitmapped formats
% which can result in "jaggedy"/blurry rendering of lines and letters as
% well as large increases in file sizes.
%
% You can find documentation about the pdfTeX application at:
% http://www.tug.org/applications/pdftex





% *** MATH PACKAGES ***
%
%\usepackage[cmex10]{amsmath}
% A popular package from the American Mathematical Society that provides
% many useful and powerful commands for dealing with mathematics. If using
% it, be sure to load this package with the cmex10 option to ensure that
% only type 1 fonts will utilized at all point sizes. Without this option,
% it is possible that some math symbols, particularly those within
% footnotes, will be rendered in bitmap form which will result in a
% document that can not be IEEE Xplore compliant!
%
% Also, note that the amsmath package sets \interdisplaylinepenalty to 10000
% thus preventing page breaks from occurring within multiline equations. Use:
%\interdisplaylinepenalty=2500
% after loading amsmath to restore such page breaks as IEEEtran.cls normally
% does. amsmath.sty is already installed on most LaTeX systems. The latest
% version and documentation can be obtained at:
% http://www.ctan.org/tex-archive/macros/latex/required/amslatex/math/





% *** SPECIALIZED LIST PACKAGES ***
%
%\usepackage{algorithmic}
% algorithmic.sty was written by Peter Williams and Rogerio Brito.
% This package provides an algorithmic environment fo describing algorithms.
% You can use the algorithmic environment in-text or within a figure
% environment to provide for a floating algorithm. Do NOT use the algorithm
% floating environment provided by algorithm.sty (by the same authors) or
% algorithm2e.sty (by Christophe Fiorio) as IEEE does not use dedicated
% algorithm float types and packages that provide these will not provide
% correct IEEE style captions. The latest version and documentation of
% algorithmic.sty can be obtained at:
% http://www.ctan.org/tex-archive/macros/latex/contrib/algorithms/
% There is also a support site at:
% http://algorithms.berlios.de/index.html
% Also of interest may be the (relatively newer and more customizable)
% algorithmicx.sty package by Szasz Janos:
% http://www.ctan.org/tex-archive/macros/latex/contrib/algorithmicx/




% *** ALIGNMENT PACKAGES ***
%
%\usepackage{array}
% Frank Mittelbach's and David Carlisle's array.sty patches and improves
% the standard LaTeX2e array and tabular environments to provide better
% appearance and additional user controls. As the default LaTeX2e table
% generation code is lacking to the point of almost being broken with
% respect to the quality of the end results, all users are strongly
% advised to use an enhanced (at the very least that provided by array.sty)
% set of table tools. array.sty is already installed on most systems. The
% latest version and documentation can be obtained at:
% http://www.ctan.org/tex-archive/macros/latex/required/tools/


%\usepackage{mdwmath}
%\usepackage{mdwtab}
% Also highly recommended is Mark Wooding's extremely powerful MDW tools,
% especially mdwmath.sty and mdwtab.sty which are used to format equations
% and tables, respectively. The MDWtools set is already installed on most
% LaTeX systems. The lastest version and documentation is available at:
% http://www.ctan.org/tex-archive/macros/latex/contrib/mdwtools/


% IEEEtran contains the IEEEeqnarray family of commands that can be used to
% generate multiline equations as well as matrices, tables, etc., of high
% quality.


%\usepackage{eqparbox}
% Also of notable interest is Scott Pakin's eqparbox package for creating
% (automatically sized) equal width boxes - aka "natural width parboxes".
% Available at:
% http://www.ctan.org/tex-archive/macros/latex/contrib/eqparbox/





% *** SUBFIGURE PACKAGES ***
%\usepackage[tight,footnotesize]{subfigure}
% subfigure.sty was written by Steven Douglas Cochran. This package makes it
% easy to put subfigures in your figures. e.g., "Figure 1a and 1b". For IEEE
% work, it is a good idea to load it with the tight package option to reduce
% the amount of white space around the subfigures. subfigure.sty is already
% installed on most LaTeX systems. The latest version and documentation can
% be obtained at:
% http://www.ctan.org/tex-archive/obsolete/macros/latex/contrib/subfigure/
% subfigure.sty has been superceeded by subfig.sty.



%\usepackage[caption=false]{caption}
%\usepackage[font=footnotesize]{subfig}
% subfig.sty, also written by Steven Douglas Cochran, is the modern
% replacement for subfigure.sty. However, subfig.sty requires and
% automatically loads Axel Sommerfeldt's caption.sty which will override
% IEEEtran.cls handling of captions and this will result in nonIEEE style
% figure/table captions. To prevent this problem, be sure and preload
% caption.sty with its "caption=false" package option. This is will preserve
% IEEEtran.cls handing of captions. Version 1.3 (2005/06/28) and later
% (recommended due to many improvements over 1.2) of subfig.sty supports
% the caption=false option directly:
%\usepackage[caption=false,font=footnotesize]{subfig}
%
% The latest version and documentation can be obtained at:
% http://www.ctan.org/tex-archive/macros/latex/contrib/subfig/
% The latest version and documentation of caption.sty can be obtained at:
% http://www.ctan.org/tex-archive/macros/latex/contrib/caption/




% *** FLOAT PACKAGES ***
%
%\usepackage{fixltx2e}
% fixltx2e, the successor to the earlier fix2col.sty, was written by
% Frank Mittelbach and David Carlisle. This package corrects a few problems
% in the LaTeX2e kernel, the most notable of which is that in current
% LaTeX2e releases, the ordering of single and double column floats is not
% guaranteed to be preserved. Thus, an unpatched LaTeX2e can allow a
% single column figure to be placed prior to an earlier double column
% figure. The latest version and documentation can be found at:
% http://www.ctan.org/tex-archive/macros/latex/base/



%\usepackage{stfloats}
% stfloats.sty was written by Sigitas Tolusis. This package gives LaTeX2e
% the ability to do double column floats at the bottom of the page as well
% as the top. (e.g., "\begin{figure*}[!b]" is not normally possible in
% LaTeX2e). It also provides a command:
%\fnbelowfloat
% to enable the placement of footnotes below bottom floats (the standard
% LaTeX2e kernel puts them above bottom floats). This is an invasive package
% which rewrites many portions of the LaTeX2e float routines. It may not work
% with other packages that modify the LaTeX2e float routines. The latest
% version and documentation can be obtained at:
% http://www.ctan.org/tex-archive/macros/latex/contrib/sttools/
% Documentation is contained in the stfloats.sty comments as well as in the
% presfull.pdf file. Do not use the stfloats baselinefloat ability as IEEE
% does not allow \baselineskip to stretch. Authors submitting work to the
% IEEE should note that IEEE rarely uses double column equations and
% that authors should try to avoid such use. Do not be tempted to use the
% cuted.sty or midfloat.sty packages (also by Sigitas Tolusis) as IEEE does
% not format its papers in such ways.


%\ifCLASSOPTIONcaptionsoff
%  \usepackage[nomarkers]{endfloat}
% \let\MYoriglatexcaption\caption
% \renewcommand{\caption}[2][\relax]{\MYoriglatexcaption[#2]{#2}}
%\fi
% endfloat.sty was written by James Darrell McCauley and Jeff Goldberg.
% This package may be useful when used in conjunction with IEEEtran.cls'
% captionsoff option. Some IEEE journals/societies require that submissions
% have lists of figures/tables at the end of the paper and that
% figures/tables without any captions are placed on a page by themselves at
% the end of the document. If needed, the draftcls IEEEtran class option or
% \CLASSINPUTbaselinestretch interface can be used to increase the line
% spacing as well. Be sure and use the nomarkers option of endfloat to
% prevent endfloat from "marking" where the figures would have been placed
% in the text. The two hack lines of code above are a slight modification of
% that suggested by in the endfloat docs (section 8.3.1) to ensure that
% the full captions always appear in the list of figures/tables - even if
% the user used the short optional argument of \caption[]{}.
% IEEE papers do not typically make use of \caption[]'s optional argument,
% so this should not be an issue. A similar trick can be used to disable
% captions of packages such as subfig.sty that lack options to turn off
% the subcaptions:
% For subfig.sty:
% \let\MYorigsubfloat\subfloat
% \renewcommand{\subfloat}[2][\relax]{\MYorigsubfloat[]{#2}}
% For subfigure.sty:
% \let\MYorigsubfigure\subfigure
% \renewcommand{\subfigure}[2][\relax]{\MYorigsubfigure[]{#2}}
% However, the above trick will not work if both optional arguments of
% the \subfloat/subfig command are used. Furthermore, there needs to be a
% description of each subfigure *somewhere* and endfloat does not add
% subfigure captions to its list of figures. Thus, the best approach is to
% avoid the use of subfigure captions (many IEEE journals avoid them anyway)
% and instead reference/explain all the subfigures within the main caption.
% The latest version of endfloat.sty and its documentation can obtained at:
% http://www.ctan.org/tex-archive/macros/latex/contrib/endfloat/
%
% The IEEEtran \ifCLASSOPTIONcaptionsoff conditional can also be used
% later in the document, say, to conditionally put the References on a
% page by themselves.





% *** PDF, URL AND HYPERLINK PACKAGES ***
%
%\usepackage{url}
% url.sty was written by Donald Arseneau. It provides better support for
% handling and breaking URLs. url.sty is already installed on most LaTeX
% systems. The latest version can be obtained at:
% http://www.ctan.org/tex-archive/macros/latex/contrib/misc/
% Read the url.sty source comments for usage information. Basically,
% \url{my_url_here}.





% *** Do not adjust lengths that control margins, column widths, etc. ***
% *** Do not use packages that alter fonts (such as pslatex).         ***
% There should be no need to do such things with IEEEtran.cls V1.6 and later.
% (Unless specifically asked to do so by the journal or conference you plan
% to submit to, of course. )


% correct bad hyphenation here
\hyphenation{op-tical net-works semi-conduc-tor}

\pagestyle{empty}

\begin{document}
%
% paper title
% can use linebreaks \\ within to get better formatting as desired
\title{Constructive Logic Approach to Solving the Unexpected Hanging Paradox}
%
%
% author names and IEEE memberships
% note positions of commas and nonbreaking spaces ( ~ ) LaTeX will not break
% a structure at a ~ so this keeps an author's name from being broken across
% two lines.
% use \thanks{} to gain access to the first footnote area
% a separate \thanks must be used for each paragraph as LaTeX2e's \thanks
% was not built to handle multiple paragraphs
%

\author{Michael~Shell,~\IEEEmembership{Member,~IEEE,}
        John~Doe,~\IEEEmembership{Fellow,~OSA,}
        and~Jane~Doe,~\IEEEmembership{Life~Fellow,~IEEE}}% <-this % stops a space

% note the % following the last \IEEEmembership and also \thanks -
% these prevent an unwanted space from occurring between the last author name
% and the end of the author line. i.e., if you had this:
%
% \author{....lastname \thanks{...} \thanks{...} }
%                     ^------------^------------^----Do not want these spaces!
%
% a space would be appended to the last name and could cause every name on that
% line to be shifted left slightly. This is one of those "LaTeX things". For
% instance, "\textbf{A} \textbf{B}" will typeset as "A B" not "AB". To get
% "AB" then you have to do: "\textbf{A}\textbf{B}"
% \thanks is no different in this regard, so shield the last } of each \thanks
% that ends a line with a % and do not let a space in before the next \thanks.
% Spaces after \IEEEmembership other than the last one are OK (and needed) as
% you are supposed to have spaces between the names. For what it is worth,
% this is a minor point as most people would not even notice if the said evil
% space somehow managed to creep in.



% The paper headers
\markboth{Journal of \LaTeX\ Class Files,~Vol.~6, No.~1, January~2007}%
{Shell \MakeLowercase{\textit{et al.}}: Bare Demo of IEEEtran.cls for Journals}
% The only time the second header will appear is for the odd numbered pages
% after the title page when using the twoside option.
%
% *** Note that you probably will NOT want to include the author's ***
% *** name in the headers of peer review papers.                   ***
% You can use \ifCLASSOPTIONpeerreview for conditional compilation here if
% you desire.




% If you want to put a publisher's ID mark on the page you can do it like
% this:
%\IEEEpubid{0000--0000/00\$00.00~\copyright~2007 IEEE}
% Remember, if you use this you must call \IEEEpubidadjcol in the second
% column for its text to clear the IEEEpubid mark.



% use for special paper notices
%\IEEEspecialpapernotice{(Invited Paper)}



\maketitle
\thispagestyle{empty}


\begin{abstract}
In this work we define a novel approach to formally specifying the
unexpected hanging paradox, sometimes called the surprise examination paradox, using a
constructive logical framework. We build this formal specification using the
Coq proof assistant.
This paradox requires the formalization of the notion of
a \emph{surprise} event, which, for the purposes of this paradox, is usually interpreted as
the inability to predict what day a specific event takes place. As in existing work,
the use of constructive logic allows us to represent knowledge as provability.
However, unlike many previous formalizations, we define a system, parametrized
by days of the week, where intuitive conclusions can be justified formally
and without inconsistency.
We are able to achieve this by making a nuanced observation about the
formal statement regarding the uniqueness of the occurence of the event, as
well as the informal meaning of surprise. We
believe that this offers a satisfying resolution to the paradox.
We also Coq-formalize an existing unsatisfactory interpretation of surprise
and compare it to ours.
\end{abstract}
% IEEEtran.cls defaults to using nonbold math in the Abstract.
% This preserves the distinction between vectors and scalars. However,
% if the journal you are submitting to favors bold math in the abstract,
% then you can use LaTeX's standard command \boldmath at the very start
% of the abstract to achieve this. Many IEEE journals frown on math
% in the abstract anyway.

% Note that keywords are not normally used for peerreview papers.
\begin{IEEEkeywords}
surprise examination, paradox, unexpected hanging, Coq, constructive logic.
\end{IEEEkeywords}






% For peer review papers, you can put extra information on the cover
% page as needed:
% \ifCLASSOPTIONpeerreview
% \begin{center} \bfseries EDICS Category: 3-BBND \end{center}
% \fi
%
% For peerreview papers, this IEEEtran command inserts a page break and
% creates the second title. It will be ignored for other modes.
\IEEEpeerreviewmaketitle



\section{Introduction}
% The very first letter is a 2 line initial drop letter followed
% by the rest of the first word in caps.
%
% form to use if the first word consists of a single letter:
% \IEEEPARstart{A}{demo} file is ....
%
% form to use if you need the single drop letter followed by
% normal text (unknown if ever used by IEEE):
% \IEEEPARstart{A}{}demo file is ....
%
% Some journals put the first two words in caps:
% \IEEEPARstart{T}{his demo} file is ....
%
% Here we have the typical use of a "T" for an initial drop letter
% and "HIS" in caps to complete the first word.
% \IEEEPARstart{T}{his} demo file is intended to serve as a ``starter file''
% for IEEE journal papers produced under \LaTeX\ using
% IEEEtran.cls version 1.7 and later.
% % You must have at least 2 lines in the paragraph with the drop letter
% % (should never be an issue)
% I wish you the best of success.
%
% \hfill mds
%
% \hfill January 11, 2007

The unexpected hanging paradox, also known as the surprise examination paradox,
is a logical paradox introduced in the Mathematical Games column of a
1963 issue of Scientific American \cite{american}. It describes the notion of a future event that
is both certain, and not possible to predict the exact day of occurence of,
formulated as follows :

\begin{myindent}
  A judge tells a condemned prisoner that he will be hanged at noon on one weekday
  in the following week but that the execution will be a surprise to the prisoner.
  He will not know the day of the hanging until the executioner knocks on his cell door at noon that day.

  Having reflected on his sentence, the prisoner draws the conclusion that he will
  escape from the hanging. His reasoning is in several parts. He begins by concluding
  that the "surprise hanging" can't be on Friday, as if he hasn't been hanged by
  Thursday, there is only one day left – and so it won't be a surprise if he's hanged on
  Friday. Since the judge's sentence stipulated that the hanging would be a surprise
  to him, he concludes it cannot occur on Friday.

  He then reasons that the surprise hanging cannot be on Thursday either, because
  Friday has already been eliminated and if he hasn't been hanged by Wednesday noon,
  the hanging must occur on Thursday, making a Thursday hanging not a surprise either.
  By similar reasoning, he concludes that the hanging can also not occur on Wednesday,
  Tuesday or Monday. Joyfully he retires to his cell confident that the hanging will
  not occur at all.

  The next week, the executioner knocks on the prisoner's door at noon on Wednesday –
  which, despite all the above, was an utter surprise to him. Everything the judge said came true.
\end{myindent}

Existing formalization efforts attempt to address questions like "how can we formally
define surprise?", "where is the flaw in the
reasoning of the prisoner?" and "was it contradictory for the prisoner to have been
hanged on Wednesday?". There is work on tackling these questions in multiple
different branches of philosophy and mathmatics, most notably a logical approach and
an epistemological one. It appears, however, that the majority of resolutions
of this paradox reach a conclusion espousing the impossibility of
defining a consistent system with a coherent definition of surprise that is not
self contradictory.

In the logical category, the approach most closely aligned with ours, where non-classical logic is used to
tackle the problem, is described in \cite{}. There are a number of attempts preceding
this one as well, \cite{} \cite{}. An epistemological approach \cite{}, while
from a different area of philosophy, is able to support reasoning that follows
a similar structure to the logical reasoning, which are compared in \cite{}.

In the work on this paradox which precedes ours \cite{} \cite{}, the use of constructive logic is
usually done via having a proof operator $\mathsf{Pr}$, which explicitly specifies
that a proposition to which it is applied is a constructive statement. In our
work, however, we assume the underlying logic to be constructive, and instead
give formal proofs that certain propositions are classical.

Another distinction between existing work and our formalization is that
we do not use modal or temporal logic (which is the approach in \cite{modal}).
Instead, we take advantage of the expressivity of the dependent typed logic of Coq directly to
achieve the formalization of surprise at different points in time,
by parametrizing the definition of the surprise proposition
by the day about which the proposition of surprise is constructed.

Because constructive logical is notoriously slippery, we chose to use a proof
assistant to take a closer, more high-assurance look at the
interplay between the seemingly simple conditions of this conundrum.
We also give an analysis of the relation of the informal meaning of surprise
to its formalization.

The contributions of this paper are as follows :

\begin{itemize}
  \item[(i)] a formal specification (in Coq) of the types and conditions of the that
  are shared among existing interpretations, using dependent types and constructive logic,
  Section \ref{sec:universal} ;
  \item[(ii)] a definition (in Coq) of surprise from a previous work, with an analysis of it
  in Section \ref{sec:one} ;
  \item[(iii)] a definition (in Coq) of surprise that we propose,
  Section \ref{sec:two};
  \item[(ii)] a formally verified proof that a Friday hanging is not a surprise, Section \ref{sec:two};
  \item[(iii)] a formally verified proof that our definition of
  surprise for this paradox is equivalent to the negation of the
  constraint that the hanging happens on exactly one specific day, Section \ref{sec:unique};
  \item[(iv)] an analysis of the unexpected hanging paradox offering a resolution without
  contradiction, Sections \ref{sec:two} \ref{sec:unique} \ref{sec:tgit} \ref{sec:friday};
\end{itemize}

For our code, see \cite{gitunexpect}.

Our definition of surprise on a day $td$ is formulated to reflect
the following natural language statement : a hanging has not occured on or before the day $td$,
and there exist at least two distinct future days on which a hanging
is possible (ie. it is not possible to disprove that a hanging occurs on either
of those days).
We argue that this accurately represents a lack of certainty about the exact day of
the hanging, but a certainty in its inevitability, and does not contradict
the other constraints of the paradox.

The crux of our paradox formalization analysis hinges on the observation that the uniqueness constraint
  is, in fact, never relevant to the formalization of the paradox --- neither in reasoning about
  any day with a possibility of future hanging (ie. on which
  a hanging has not yet happened), nor about a day on which it has already happened.
  We go on to argue that the constraint that a hanging necessarily happens is
   also contradictory to our definition of surprise. Moreover, the looser constraint
  from previous work,
  saying that we \emph{should not be able to disprove that a hanging happens on one of the
  days} is insufficiently strong, and is implied by our definition. However, the intuitive conclusion normally draw from it
  (that if a hanging is only possible Friday, it must happen then)
  is too strong.


\section{Coq and the Paradox}
\label{sec:form}

Coq is a proof assistant that offers a dependently typed formal language \cite{coq}.
It is capable of verifying formal user-defined proofs of propositions, as well as has support
for automation of certain kinds of proofs. The choice of Coq, as opposed to another
proof assistant such as Agda \cite{agda}, was based largely on the authors' familiarity with the system,
as any dependently typed proof verifier would serve just as well for the purposes of this formalization.

To formalize the paradox, we need to reason about days of the week on which
the paradox could happen, so we
begin by constructing a type {\tt weekDay} the terms of which are week days :

\begin{lstlisting}
  Inductive weekDay : Type :=
    | monday : weekDay
    | tuesday : weekDay
    | wednesday : weekDay
    | thursday : weekDay
    | friday : weekDay.
\end{lstlisting}

We also define a type {\tt weekAndBefore}, which represents all the weekdays in
the type above, plus the Sunday that comes before --- the purpose of this type is to
represent all the days on which one can consider the possibility of surprise,
differentiating it from the subset of days on which the hanging can occur. We
also define the two comparison functions,

\begin{lstlisting}
  isBefore, isOnOrAfter (td : weekAndBefore) (d : weekDay) : Prop := ...
\end{lstlisting}

which compute whether a given {\tt td} is before (is on or after, respectively) {\tt d},
following real-life weekday logic, eg. Sunday is before Monday.
Both of these are classical comparisons, which we prove in our code.
From here on, we use the notation $<$ for {\tt isBefore}, and
$\geq$ for {\tt isOnOrAfter}.

We do not know what day the hanging happens on, but we can specify the type of a
function that, given a day of the week, returns $\mathsf{True}$ if we can prove the hanging
happened, and $\mathsf{False}$ if we can prove it did not. We reason about this function
in the presence of preconditions that are formal interpretations of those described
in the paradox. We leave this function as a variable :

\begin{lstlisting}[mathescape=true]
  Variable hangingOnDay : weekDay $\to$ Prop.
\end{lstlisting}

We discuss the existence of such a predicate that
works with in definition of surprise as part of future work.
Next, we define a predicate that formalizes the notion that no hanging has occured
yet (up to and including the parameter day {\tt td}, representing \emph{today}) :

\begin{lstlisting}[mathescape=true]
  Definition noHangingYet
      (td : weekAndBefore) :=
    $\forall$ d, td $\geq$ d
    $\to$ $\neg $ hangingOnDay d.
\end{lstlisting}

This says that for any day {\tt d}, if it is before today {\tt td}, no hanging
happened on {\tt d}.

We use the double negation {\tt $\neg \neg$ hangingOnDay d} to formalize the statement that it is
not possible to disprove that a hanging occurs on day {\tt d}. That is, a hanging
is \emph{possible} on a given day.

\section{Surprise parametrization and universal conditions}
\label{sec:universal}

Notice that the {\tt noHangingYet} function we introduced is actually a
dependent proposition (a predicate), parametrized by
the day {\tt td} that is \emph{today}. This is because we are interested in
being able to reason under the conditions of the past being defined, but
would like to vary what "the past" refers to. Similarly, we parametrize the
constraints of the paradox so that we are able to reason about
whether we are still within them of at each point in the week,
which is the level of detail in we are inrested in. This lets us, for example,
reason about the impossibility of surprise on Thursday, see Section \ref{sec:tgit}.

We now specify two conditions, implicit in the informal
description of the paradox, that are shared across
many formal interpretations including \cite{} \cite{},  and ones discussed here :

\begin{itemize}
  \item[(i)] if today is Friday, ie. the whole
  week has passed, we know that a hanging necessarily happened
  \begin{lstlisting}[mathescape=true]
    td = someWeekDay friday
      $\to$ $\exists$ d, hangingOnDay d
  \end{lstlisting}
  \item[(ii)] if a hanging has occured, it is unique
  \begin{lstlisting}[mathescape=true]
    ($\exists$ d, hangingOnDay d)
      $\to$ uniqueHanging dayBefore
  \end{lstlisting}
\end{itemize}

where {\tt uniqueHanging} is a predicate formalizing that after a given day {\tt td},
there can be at most one day on which a hanging occurs, and thus,
{\tt uniqueHanging dayBefore} states this about the entire week.

\begin{lstlisting}[mathescape=true]
  Definition uniqueHanging
      (td : weekAndBefore) :
    $\forall$ d d',
    td $<$ d $\wedge$ td $<$ d',
    hangingOnDay d $\to$
    hangingOnDay d' $\to$
    d = d'.
\end{lstlisting}

We note here that the preconditions in both implications are such that if
they are satisfied, intuitively surprise should not be possible : (i)
requires that the week is already over, and (ii) requires that a hanging
has happened. The time at which we may be surprised by a hanging is strictly
before it happens and before the week is over, so neither of these should
actually play any role in the reasoning about surprise, which is reflected
in our upcoming definition of surprise. However, (i) and (ii) together express all the
other constraints of the paradox besides the one requiring the hanging to
be a surprise (now we just need to add that part!).

Surprise requires that a future hanging be possible - on more than zero
of the remaining weekdays after today.
So, the constraint that at least one future possible hanging day exists is
clearly necessary. However, the case for at least one possible day not
being sufficient to define surprise is a bit more nuanced.

\section{At least one possible day}.
\label{sec:one}

We specify this interpretation
of the conditions of the paradox for each possible day as:

\begin{lstlisting}[mathescape=true]
  Definition onePossiblePRDX (td : weekAndBefore)  :=
    (td = someWeekDay friday
      $\to$ $\exists$ d, hangingOnDay d)
    $\wedge$
    (($\exists$ d, hangingOnDay d)
      $\to$ uniqueHanging dayBefore)
    $\wedge$
    (td $<$ friday $\wedge$ noHangingYet td) ->
      exists d, td $<$ d $\wedge$ $\neg~\neg~$ hangingOnDay d).
\end{lstlisting}

where the first two conjuncts are as described above, and
the third one corresponds to "if today is not
  yet Friday, there is a possible day on which a hanging may happen", which
  we refer to as the {\tt onePossible} definition of surprise.

This definition is one of the alternatives discussed in \cite{}. No inconsistency
is introduced here, in fact, the hanging can still be a surprise even if it
happens on a Friday! The intuition behind this is : if no hanging happened by
Thursday, it is still only possible to prove {\tt $\neg~\neg$~hangingOnDay friday},
from which we are not able to deduce that {\tt hangingOnDay friday}.

Note here that including the constraint that \emph{a hanging must happen
by the end of the week} is still not sufficient to conclude certainty
from unique possibility. The reason for this is that the conclusion that
a hanging definitely happened on one of the week days can only be made
after Friday has come. The possibility of surprise, meanwhile, has the
precondition that the whole week has not passed yet (ie. Friday has not yet come).
These preconditions are mutually exclusive.

The condition that if a hanging has happened, it must be unique, also does not
give us any reasoning power, as we are not able to conclude {\tt hangingOnDay friday}.

Let us consider what happens if we impose an additional constraint stating that, assuming that
(i) there is a possible hanging day, and (ii) a hanging must be unique if it has
happened, exactly one \emph{possible} hanging
day implies that it \emph{necessarily happens} on that day. This is also explored in
\cite{}, with a similar conclusion to the one we draw here. The following
proposition states that there is a possible hanging day, and that a possible
unique day implies certainty of hanging on that day :

\begin{lstlisting}[mathescape=true]
  Definition existsUniqueHappens :=
    $\exists$ d, ($\neg~\neg$ hangingOnDay d
    $\wedge$
    ($\forall$ d',
      $\neg~\neg$ hangingOnDay d
      $\to$ $\neg~\neg$ hangingOnDay d'
      $\to$ d = d') $\to$
      (hangingOnDay d)).
\end{lstlisting}

Now, the following statement expresses that {\tt existsUniqueHappens} lets us
conclude that {\tt hangingOnDay} must then be classical (the proof is in the
associated code) :

\begin{lstlisting}[mathescape=true]
  Lemma euhImpClassical :
    (uniqueHanging dayBefore) $\to$
    (exists d, ~~ hangingOnDay d) $\to$
    existsUniqueHappens $\to$
    ($\forall$ d,
    $\neg~$ hangingOnDay d $\vee$ hangingOnDay d).
\end{lstlisting}

This proposition, without justification, is the crux of the reasoning the prisoner
uses to informally to arrive at the judgement that
if a hanging hasn't happened by Thursday, it must happen on Friday.
Note here that the inductive reasoning in which the prisoner engages to conclude that the
hanging cannot ever be a surprise is, in some sense, superfluous --- we can use
constructive logic reasoning to prove, without induction, that "if we can conclude
from existence plus uniqueness of a possible hanging day, that it is certain on that day,
\emph{our judgement about hanging occuring on any day must necessarily be classical}".
The proof makes use of the fact that the equality comparison {\tt d = d'} is
classical.

The intuition seems to be correct in making the assumption that a unique possibility
implies certainty, and the logic leading to this inconsistency with the informal
definition appears solid.
However, we will see that the problem lies in the attempt
to allow the possibility of, and draw conclusions from, having a \emph{unique possible }
hanging day.

This definition leaves us with the following conclusions about defining surprise as
having at least one possible hanging day :

\begin{itemize}
  \item[(i)] such a definition of surprise is not strong enough to allow us to conclude
  {\tt existsUniqueHappens}, in particular, that a hanging
  must happen Friday given that it has not occured by Thursday, and is
  therefore not a surprise in that case ; and \newline
  \item[(ii)] if we \emph{were} to be able to conclude {\tt existsUniqueHappens},
  reasoning about surprise
  becomes classical, so we can always figure out the hanging day in advance.
\end{itemize}

Both possibilities appear problematic : (i) does not allow us to make
a conclusion that we would like make,
and (ii) does not support reasoning non-classically, which removes any ambiguity
about the future hanging day, and therefore, the possibility of surprise.
Let us see why adding a different additional constraint (ie. other than deriving
certainty from a unique possibility) will help.

\section{Full paradox : at least two possible days}.
\label{sec:two}

The way we propose to strengthen the conditions of surprise is by increasing
the number of possible days required for surprise to two :

\begin{lstlisting}[mathescape=true]
  Definition twoPossible
      (td : weekAndBefore) :=
    $\exists$ d d', d $\neq$ d'
    $\wedge$ $\neg \neg$ hangingOnDay d
    $\wedge$ $\neg \neg$ hangingOnDay d'
    $\wedge$ td $<$ d $\wedge$ td $<$ d'.
\end{lstlisting}

We can read this as follows :

\begin{myindent}
  There exist two distinct days after the day {\tt td} such that
  a hanging is possible (ie. cannot disprove that it happens then)
\end{myindent}

Intuitively, this constraint makes sense --- if I am not sure what day something
will occur, there must be at least two possible future days on which it could occur,
as expressed in {\tt twoPossible}.
To formulate the complete conditions of the paradox for each day, we formalize that, in addition
to the two constraints justified earlier, we include the constraint that
"if today is before Friday, and no hanging has yet happened, there are at least
two possible distinct hanging days in the future" :

\begin{lstlisting}[mathescape=true]
  Definition twoPossiblePRDX
      (td : weekAndBefore)  :=
    (td = someWeekDay friday
      $\to$ $\exists$ d, hangingOnDay d)
    $\wedge$
    (($\exists$ d, hangingOnDay d)
      $\to$ uniqueHanging dayBefore)
    $\wedge$
    (td $<$ friday $\wedge$ noHangingYet td) ->
      twoPossible td.
\end{lstlisting}

Now, the first thing we can formally conclude about this definition is that it indeed rules out
a Friday hanging. The following lemma says that it is not possible that by Thursday,
no hanging has happened, but there are still two distinct possible days for it to
happen in the future.

\begin{lstlisting}[mathescape=true]
  Lemma cantBeSurpFriday :
    twoPossiblePRDX (someWeekDay thursday)
    $\to$ False.
\end{lstlisting}

The proof (see code) is trivial, since there is only one day (Friday) left in the week,
and no hangings are possible on past days.

This reasoning does not work, however, for any
day before Thursday, since there are (for any day before Thursday) at least two future
days about which we have no
data contradicting the possibility of a hanging.
Inductive reasoning in attempt to conclude that a Thursday hanging is
predictable on Wednesday does not work because we do not have enough data on Wendesday
to conclude whether a hanging will be Thursday or Friday. That is, to discount Friday
as a possibility of surprise hanging, we \emph{need to know that there was no
hanging Thursday}.

So, this definition of surprise aligns with our intuition by making surprise
by a Friday hanging possible, but does not appear to contradict the possibility
on any other day. It has, however, an unintuitive feature.

\section{Negation of uniqueness. }
\label{sec:unique}

With this definition of surprise, surprise is never possible when there is
exactly one possible hanging day. The constraint, {\tt uniqueHanging}, that
a (provable) hanging day is unique is, in fact, equivalent to the constraint
that the (possible) hanging day must be unique,

\begin{lstlisting}[mathescape=true]
  Definition uniqueMaybe
      (td : weekAndBefore) :
    $\forall$ d d',
    td $<$ d $\wedge$ td $<$ d',
    $\neg~\neg$ hangingOnDay d $\to$
    $\neg~\neg$ hangingOnDay d' $\to$
    d = d'.
\end{lstlisting}

We parametrize the above propositions by $td$ to express that they apply to
a specific subset of the week days --- those days that are after today,
eg., {\tt uniqueMaybe dayBefore} specifies that if a hanging is provable to be
on \emph{any two weekdays}, those days must be the same.
We prove that they the two propositions are equivalent :

\begin{lstlisting}[mathescape=true]
  Lemma uniqueMaybeEqv
      (td : weekAndBefore) :
    uniqueHanging td
    $\leftrightarrow$
    uniqueMaybe td.
\end{lstlisting}

The proof of this relies on, again, the fact that $d = d'$ is a classical
comparison. Moreover, the {\tt twoPossible} constraint
in our definition of surprise \emph{is the negation of} {\tt uniqueHanging}
(and therefore also {\tt uniqueMaybe}) :

\begin{lstlisting}[mathescape=true]
  Lemma twoNotUnique :
    $\forall$ td,
    $\neg$ uniqueHanging td
    $\leftrightarrow$
    twoPossible td.
\end{lstlisting}

We make an observations about this : a future possible day of the hanging is
necessarily not unique. This seems wrong --- we would expect a hanging to be
unique, it's implicit in the
description of the paradox. Let us inspect this closer, however.
Our conditions of this paradox on day $td$ are that either
the week is over and the hanging has occured, or, for $td \neq thrusday, friday$ surprise is
only possible if the hanging has not yet happened. We not that

Once a hanging occurs, we can no longer reason about it having been a surprise
That is, a judgement of surprise can only made about
an event in the context of a lack of information about it (in this case,
whether a hanging will happen on a specific future day).

This is perhaps the most paradoxical idea here. Any reasoning that includes
in its hypotheses both that a hanging happened after {\tt td} and that it is a surprise
(in our case, specified as {\tt twoPossible td}) is intuitively meaningless. This is not a
formal statement, but rather a statement about what surprise means. You cannot
be surprised by an event that has already occured.

As a sidenote, this is a nuance of how we use language. One can reason about a surprising past
event from the perspective of a time prior to that event, but to do so,
the available hypotheses must only include information known at that time, ie.
not the occurence of that event. This is why statements such as "I was surprised
by a hanging when it happened", or "last week I was surprised when I found out
that it had happened to my friend two weeks ago", make sense, and are usually
what is meant when one says "I am surprised by a past hanging".

Now, back to why it is ok for hanging day to not to be required to be unique
before the hanging occurs : once the (first) hanging
occurs on some {\tt d}, we now have {\tt $\neg~$ noHangingYet (someWeekDay d)}.
That means that the {\tt twoPossible (someWeekDay d)} implication of {\tt noHangingYet (someWeekDay d)}
is no longer required to hold, and also that {\tt exists d, hangingOnDay d}, so
its consequence {\tt uniqueHanging dayBefore} is now required to hold.
So, uniqueness constraint "kicks in" after the hanging first happens.

\section{TGI Thursday}
\label{sec:tgit}

With this definition of the paradox, surprise is already not possible on Thursday.
The reason for this is not that the definition allows us to conclude that a hanging will
definitely happen on Friday once we get to Thursday. Rather, it is that the
conditions for surprise are not satisfied on Thursday in another way (a missing
a possible day).

We have already discussed that, since the definition
is parametrized by the day on which we judge whether the paradox conditions
are met, surprise (and all other conditions of the paradox) actually form
a different proposition for each day. And so, an impossibility of surprise
Thursday does not affect what we can prove about the rest of the days.
With that in mind, we can choose to make a special case
for the paradox characterization, which only applies when exactly one day remains as a possibility
for a hanging --- and there is only one such day, Thursday.

The special case to add is exactly the conclusion we rejected in the {\tt onePossible}
version of surprise : {\tt existsUniqueHappens}. Note here that
the difference between one- and two-possible versions of surprise, with respect
to adding this constraint,
is that in the two-case, we use the paradox definition to conclude that Thursday
does not allow surprise. In the one-case, on the other hand, we \emph{rely} on the reasoning in
{\tt existsUniqueHappens} to conclude surprise is not possible on Thursday,
which leads an unsavoury conclusion.

We do not add it, after all, even though the reasoning it allows us to do
appears coherent. The reason for this is that the preconditions of this
constraint already conflict with those of the {\tt twoPossible} surprise definition, as well
as with the preconditions of the universal paradox constraints. So, it does not
add additional reasoning power to our definition.

\section{The week is done }
\label{sec:friday}

In both versions of the definition of surprise ({\tt twoPossiblePRDX} and
{\tt onePossiblePRDX}), Friday is explicitly excluded from the surprise
part of the paradox definition.
We exclude this day since Friday arriving means that the
week has passed and there is no possible way for a hanging to occur. We also
conclude that the hanging must have already occured.

\section{Conclusion and Future Work}

We have presented a formalization of the unexpected hanging paradox that appears to capture the
conditions of the paradox, meanwhile both aligning with the
intuition about Friday, and not reasoning ourselves out of the hanging entirely.
We formalized this definition, along with some related proofs, in the Coq
proof assistant.
A few key ideas were needed to achieve this.

The starting point was the observation that constructive logic
is required to represent knowledge in order to allow for the possibility that we
may not know something to be either true or false --- such as whether a hanging
occured on a given day. Next, we observe that surprise is a different notion
on each of the days, so our definition is parametrized by the day on which
we reason about its possibility. Then, we notice that the universal conditions of the
paradox only apply when surprise is intuitively not possible (ie. the end of the
week or after the hanging), and we separate these out from the surprise specification.

Finally, we formalize and compare two definitions of surprise. The first,
{\tt onePossible}, has been exlored in prior work. We discuss why it has
consequences contradictory to our intuition about the paradox, and attempts to
rectify them in the intuitive way result in the
exact undesirable conclusion that the prisoner makes in the paradox description.
The second is {\tt twoPossible}, which we proposed, and it is a strengthening of
the first definition that does not cause the reasoning about the future hanging to
become classical (unlike {\tt onePossible}).

The surprise formalization we propose drops (in fact, negates) the constraint
guaranteeing the uniqueness of the future hanging. The subtlety here is in the idea that the
constraint still applies \emph{once the hanging happens}, but multiple
distinct days for a future hanging are still required up until then.
Our Coq formalization of the paradox stands out from existing work in (i)
its mechanization of the problem (ii) providing a resolution to the paradox that aligns with
intuition (iii) leading us to making new subtle observations about the
constraints of the problem and how they are hidden by language.

As part of future work, a proof of the existence of {\tt hangingOnDay}, such that
a version of {\tt twoPossiblePRDX td} (parametrized by the hanging decision
function) holds for all days except Thursday,

\begin{lstlisting}[mathescape=true]
  Proposition existsHangFunc :
    $\exists$ hangingOn, $\forall$ td,
    twoPossiblePRDX_param
      hangingOn td.
\end{lstlisting}

would valuable in
establishing our formalization as the universally accepted one. Additionally,
this paradox formalization could be further analyzed by way of considering its
relationship to the axiom of choice. This is due to its (at least surface level)
resemblance to the way the AC makes a connection between classical logic
and a choice function \cite{} as well as arbitrary elements \cite{}.

Another future direction we consider is generalizing the surprise hanging
approach we propose to using constructive logic for describing and proving
the existence of a function {\tt myPick}
which represents choosing an arbitrary value from a decidable set. In particular,
given a decidable set {\tt W}, \newline

\begin{myindent}
For any subset {\tt S $\subseteq$ W} of cardinality at least 2, such that for all
{\tt s $\in$ W - S}, $\neg ${\tt myPick s}, there exist at least two distinct
elements in {\tt S}, such that $\neg~\neg~${\tt myPick s}.
\end{myindent}


% An example of a floating figure using the graphicx package.
% Note that \label must occur AFTER (or within) \caption.
% For figures, \caption should occur after the \includegraphics.
% Note that IEEEtran v1.7 and later has special internal code that
% is designed to preserve the operation of \label within \caption
% even when the captionsoff option is in effect. However, because
% of issues like this, it may be the safest practice to put all your
% \label just after \caption rather than within \caption{}.
%
% Reminder: the "draftcls" or "draftclsnofoot", not "draft", class
% option should be used if it is desired that the figures are to be
% displayed while in draft mode.
%
%\begin{figure}[!t]
%\centering
%\includegraphics[width=2.5in]{myfigure}
% where an .eps filename suffix will be assumed under latex,
% and a .pdf suffix will be assumed for pdflatex; or what has been declared
% via \DeclareGraphicsExtensions.
%\caption{Simulation Results}
%\label{fig_sim}
%\end{figure}

% Note that IEEE typically puts floats only at the top, even when this
% results in a large percentage of a column being occupied by floats.


% An example of a double column floating figure using two subfigures.
% (The subfig.sty package must be loaded for this to work.)
% The subfigure \label commands are set within each subfloat command, the
% \label for the overall figure must come after \caption.
% \hfil must be used as a separator to get equal spacing.
% The subfigure.sty package works much the same way, except \subfigure is
% used instead of \subfloat.
%
%\begin{figure*}[!t]
%\centerline{\subfloat[Case I]\includegraphics[width=2.5in]{subfigcase1}%
%\label{fig_first_case}}
%\hfil
%\subfloat[Case II]{\includegraphics[width=2.5in]{subfigcase2}%
%\label{fig_second_case}}}
%\caption{Simulation results}
%\label{fig_sim}
%\end{figure*}
%
% Note that often IEEE papers with subfigures do not employ subfigure
% captions (using the optional argument to \subfloat), but instead will
% reference/describe all of them (a), (b), etc., within the main caption.


% An example of a floating table. Note that, for IEEE style tables, the
% \caption command should come BEFORE the table. Table text will default to
% \footnotesize as IEEE normally uses this smaller font for tables.
% The \label must come after \caption as always.
%
%\begin{table}[!t]
%% increase table row spacing, adjust to taste
%\renewcommand{\arraystretch}{1.3}
% if using array.sty, it might be a good idea to tweak the value of
% \extrarowheight as needed to properly center the text within the cells
%\caption{An Example of a Table}
%\label{table_example}
%\centering
%% Some packages, such as MDW tools, offer better commands for making tables
%% than the plain LaTeX2e tabular which is used here.
%\begin{tabular}{|c||c|}
%\hline
%One & Two\\
%\hline
%Three & Four\\
%\hline
%\end{tabular}
%\end{table}


% Note that IEEE does not put floats in the very first column - or typically
% anywhere on the first page for that matter. Also, in-text middle ("here")
% positioning is not used. Most IEEE journals use top floats exclusively.
% Note that, LaTeX2e, unlike IEEE journals, places footnotes above bottom
% floats. This can be corrected via the \fnbelowfloat command of the
% stfloats package.




% if have a single appendix:
%\appendix[Proof of the Zonklar Equations]
% or
%\appendix  % for no appendix heading
% do not use \section anymore after \appendix, only \section*
% is possibly needed

% use appendices with more than one appendix
% then use \section to start each appendix
% you must declare a \section before using any
% \subsection or using \label (\appendices by itself
% starts a section numbered zero.)
%


% Can use something like this to put references on a page
% by themselves when using endfloat and the captionsoff option.
\ifCLASSOPTIONcaptionsoff
  \newpage
\fi



% trigger a \newpage just before the given reference
% number - used to balance the columns on the last page
% adjust value as needed - may need to be readjusted if
% the document is modified later
%\IEEEtriggeratref{8}
% The "triggered" command can be changed if desired:
%\IEEEtriggercmd{\enlargethispage{-5in}}

% references section

% can use a bibliography generated by BibTeX as a .bbl file
% BibTeX documentation can be easily obtained at:
% http://www.ctan.org/tex-archive/biblio/bibtex/contrib/doc/
% The IEEEtran BibTeX style support page is at:
% http://www.michaelshell.org/tex/ieeetran/bibtex/
%\bibliographystyle{IEEEtran}
% argument is your BibTeX string definitions and bibliography database(s)
%\bibliography{IEEEabrv,../bib/paper}
%
% <OR> manually copy in the resultant .bbl file
% set second argument of \begin to the number of references
% (used to reserve space for the reference number labels box)
\begin{thebibliography}{1}

\bibitem{IEEEhowto:kopka}
H.~Kopka and P.~W. Daly, \emph{A Guide to \LaTeX}, 3rd~ed.\hskip 1em plus
  0.5em minus 0.4em\relax Harlow, England: Addison-Wesley, 1999.

\end{thebibliography}

% biography section
%
% If you have an EPS/PDF photo (graphicx package needed) extra braces are
% needed around the contents of the optional argument to biography to prevent
% the LaTeX parser from getting confused when it sees the complicated
% \includegraphics command within an optional argument. (You could create
% your own custom macro containing the \includegraphics command to make things
% simpler here.)
%\begin{biography}[{\includegraphics[width=1in,height=1.25in,clip,keepaspectratio]{mshell}}]{Michael Shell}
% or if you just want to reserve a space for a photo:
%
% \begin{IEEEbiography}{Michael Shell}
% Biography text here.
% \end{IEEEbiography}
%
% % if you will not have a photo at all:
% \begin{IEEEbiographynophoto}{John Doe}
% Biography text here.
% \end{IEEEbiographynophoto}
%
% % insert where needed to balance the two columns on the last page with
% % biographies
% %\newpage
%
% \begin{IEEEbiographynophoto}{Jane Doe}
% Biography text here.
% \end{IEEEbiographynophoto}

% You can push biographies down or up by placing
% a \vfill before or after them. The appropriate
% use of \vfill depends on what kind of text is
% on the last page and whether or not the columns
% are being equalized.

%\vfill

% Can be used to pull up biographies so that the bottom of the last one
% is flush with the other column.
%\enlargethispage{-5in}



% that's all folks
\end{document}
